\newpage
\chapter*{Résumé}

Depuis la découverte du boson de Higgs en 2012, la plupart de ses propriétés telles que sa masse, son spin, sa section efficace de production et son couplage aux fermions et bosons ont été mesurées. Cependant, la composante trilinéaire de l'auto-couplage du boson de Higgs $\lambda_{HHH}$ n'a pas encore été mesuré. Ce paramètre contrôle la forme du potentiel de Higgs, donnant de l'importance à sa mesure. Une déviation par rapport à la valeur prédite par le Modèle Standard (MS) indiquerait la présence d'une nouvelle physique au-delà du MS. Les déviations sont quantifiées via le modificateur \kl. Au LHC, il est mesuré à travers le taux de production de paires de boson de Higgs (HH) qui est un processus rare et aussi le seul moyen direct d'y accéder. Cette thèse présente la recherche de la production de paires de boson de Higgs dans le canal de désintégration $H(\to\gamma\gamma)H(\to\bar{b}b)$. 
Le travail est effectué avec les données de collisions de protons du Run-2 enregistrées à une énergie de $\sqrt{s} = $ 13 TeV par le détecteur ATLAS au LHC entre 2015 et 2018. Cette quantité de données correspond à une luminosité intégrée de 139 \ifb. Malgré le faible nombre d'événements signaux, le canal \bbyy bénéficie d'une signature expérimentale très propre grâce d'une excellente résolution de masse \myy, d'un fond lisse et d'un grand rapport de branchement H$\to\bar{b}b$. C'est un canal excellent pour effectuer la mesure de \kl. \\
Les photons sont tellement cruciaux pour ce canal qu'un travail d'amélioration de leur identification est présenté dans cette thèse. Un nouvel algorithme basé sur les réseaux de neurones à convolution (CNN) est implémenté pour améliorer l'efficacité de l'identification. L'algorithme CNN utilise des images des cellules du calorimètre électromagnétique pour apprendre à reconnaitre les gerbes électromagnétiques des photons et les séparer des photons provenant de la fragmentation des jets ou des faux photons. De plus, la modélisation du développement des gerbes électromagnétiques en simulation est corrigée en répartissant l'énergie entre les cellules du calorimètre électromagnétique pour se rapprocher des données réelles. \\
La sensibilité au signal \HHyybb est améliorée en appliquant une méthode d'étalonnage spécifique aux jets de $b$. Le signal est ensuite séparé des bruits de fonds (boson de Higgs et continus $\gamma\gamma$+jets) grâce à une technique d'analyse multivariée (MVA). Des outils supplémentaires sont développés pour améliorer la sensibilité de l'analyse. Ils ne sont pas inclus dans la publication de la collaboration ATLAS, mais sont disponible pour la prochaine analyse.  Aucun excès significatif d'événements \HHyybb n'est observé dans les données par rapport au bruit de fond attendu. Ainsi, une limite supérieure sur la section efficace HH est fixée. La limite observée (attendue) à un niveau de confiance (CL) de 95 \% correspond à 4,1 (5,5) fois la section efficace prévue par le MS. La limite de 95 \% CL observée (attendue) sur le modificateur \kl est égale à [-1,5 ; 6,7] ([-2,4 ; 7,7]). \\
Une extrapolation de ce résultat au Run-3 et au LHC Haute-Luminosité (HL-LHC) est également présentée dans cette thèse.