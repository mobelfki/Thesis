\newpage
\chapter*{Abstract}
Since the discovery of Higgs boson in 2012 to now, most of its properties such as its mass, spin, production cross-section and its coupling to fermions and bosons have been measured. However, the Higgs boson trilinear coupling "self-coupling" $\lambda_{HHH}$ has not been measured yet. This parameter controls the shape of the Higgs potential, explaining the importance of its measurement. Deviation from its Standard Model (SM) predicted value would indicate new physics beyond the SM (BSM). Deviations are quantified through the \kl modifier. At the LHC, it is measured through the rate of the rare Higgs pair production (HH) process, which is the only direct way to access it. This thesis presents the search for Higgs pair production in its $H(\to\gamma\gamma)H(\to\bar{b}b)$ decay channel. The search is preformed with the full Run-2 proton-proton collision data collected with the ATLAS detector at the LHC between 2015 and 2018 at a centre-of-mass energy of $\sqrt{s} = $ 13 TeV. This amount of data correspond to an integrated luminosity of 139 \ifb. Despite a low statistics, the \bbyy channel profits from a very clean experimental signature, an excellent \myy mass resolution and a smooth background and a large H$\to\bar{b}b$ branching ratio. It is therefore an excellent channel to perform \kl measurement. \\
Photon objects are so crucial for this channel, that a work concerning the improvement of photon identification is presented in this thesis. A new algorithm based on Convolutional Neural Network (CNN) is implemented to improve the photon identification efficiency. The CNN algorithm uses images from the electromagnetic calorimeter layers to learn the shower development of prompt photons and separates them from photons originating from jet fragmentation or fake photons. In addition, the mis-modelling of the shower shape in simulation is corrected by redistributing the energy between the cluster cells in the simulation to become consistent with the real data. \\
The sensitivity to \HHyybb signal is improved using a specific $b$-jet calibration method and the signal is separated from the single Higgs and the continuum $\gamma\gamma$+jets backgrounds through a Multivariate Analysis (MVA) technique. Additional tools are developed to improve the analysis sensitivity, but not used and kept to next analysis. No significant excess of \HHyybb events is observed in the data with respect to the expected background. Thus, an upper limit on the HH cross-section is set. The observed (expected) 95\% confidence level (CL) limit corresponds to 4.1 (5.5) times the cross-section predicted by the SM. The observed (expected) 95\% CL limit on the \kl modifier is also derived to be [-1.5, 6.7] ([-2.4, 7.7]).  \\
An extrapolation of this result to Run-3 and High-Luminosity LHC (HL-LHC) is also presented in this thesis.
 
\newpage
 \chapter*{Résumé}
 

Depuis la découverte du boson de Higgs en 2012, la plupart de ses propriétés telles que sa masse, son spin, sa section efficace de production et son couplage aux fermions et bosons ont été mesurées. Cependant, le couplage trilinéaire du boson de Higgs "auto-couplage" $\lambda_{HHH}$ n'a pas encore été mesuré. Ce paramètre contrôle la forme du potentiel de Higgs, expliquant l'importance de sa mesure. Une déviation par rapport à sa valeur prédite par le Modèle Standard (MS) indiquerait une nouvelle physique au-delà du MS. Les déviations sont quantifiées via le modificateur \kl. Au LHC, il est mesuré par le taux du processus rare de production de paires de Higgs (HH), qui est le seul moyen direct d'y accéder. Cette thèse présente la recherche de la production de paires de Higgs dans son canal de désintégration $H(\to\gamma\gamma)H(\to\bar{b}b)$. Le travail présenté dans cette thèse est effectuée avec les données de collisions de protons de Run-2 enregistrées à une énergie de $\sqrt{s} = $ 13 TeV par le détecteur ATLAS au LHC entre 2015 et 2018. Cette quantité de données correspond à une luminosité intégrée de 139 \ifb. Malgré de faibles statistiques, le canal \bbyy bénéficie d'une signature expérimentale très propre, d'une excellente résolution de masse \myy et d'un fond lisse et d'un grand rapport de branchement H$\to\bar{b}b$. C'est un canal excellent pour effectuer une mesure \kl. \\
Les photons sont tellement cruciaux pour ce canal, qu'un travail concernant l'amélioration de l'identification des photons est présenté dans cette thèse. Un nouvel algorithme basé sur les réseaux de neurones à convolution (CNN) est implémenté pour améliorer l'efficacité de l'identification des photons. L'algorithme CNN utilise des images des couches du calorimètre électromagnétique pour apprendre le développement des gerbes électromagnétiques des photons et les sépare des photons provenant de la fragmentation des jets ou des faux photons. De plus, la précision de la modélisation du développement des gerbes électromagnétiques en simulation est corrigée en redistribuant l'énergie entre les cellules du calorimètre électromagnétique dans la simulation pour devenir cohérente avec les données réelles. \\
La sensibilité au signal \HHyybb est améliorée en utilisant une méthode d'étalonnage spécifique $b$ -jet et le signal est séparé des bruits de fonds : Higgs et continus $\gamma\gamma$+jets grâce à une technique d'analyse multivariée (MVA). Des outils supplémentaires sont développés pour améliorer la sensibilité de l'analyse, mais ils ne sont pas utilisés et conservés pour la prochaine analyse. Aucun excès significatif d'événements \HHyybb n'est observé dans les données par rapport au bruit de fond attendu. Ainsi, une limite supérieure sur la section efficace HH est fixée. La limite observée (attendue) a un niveau de confiance (CL) de 95 \% correspond à 4.1 (5.5) fois la section efficace prévue par le SM. La limite de 95 \% CL observée (attendue) sur le modificateur \kl est également dérivée de [-1.5, 6.7] ([-2.4, 7.7]). \\
Une extrapolation de ce résultat à Run-3 et Haute-Luminosité LHC (HL-LHC) est également présentée dans cette thèse.

