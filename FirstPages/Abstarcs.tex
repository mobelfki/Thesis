\newpage
\chapter*{Abstract}
From the discovery of the Higgs boson in 2012, most of its properties such as mass, spin, production cross-section and its coupling to fermions and bosons have been measured. However, The trilinear self-coupling $\lambda_{HHH}$ of the Higgs boson has not been measured yet. This parameter controls the shape of the Higgs potential, explaining the importance of its measurement. Deviation from its Standard Model (SM) predicted value would indicate new physics beyond the SM (BSM). Deviations are quantified through the \kl modifier. At the LHC, it is measured through the rate of the rare Higgs pair production (HH) process, which is the only direct way to access it. This thesis presents the search for Higgs pair production in its $H(\to\gamma\gamma)H(\to\bar{b}b)$ decay channel. The search is performed with the full Run-2 proton-proton collision data collected with the ATLAS detector at the LHC between 2015 and 2018 at a centre-of-mass energy of $\sqrt{s} = $ 13 TeV. This amount of data corresponding to an integrated luminosity of 139 \ifb. Despite low statistics, the \bbyy channel profits from a very clean experimental signature, an excellent \myy mass resolution, a smooth background and a large H$\to\bar{b}b$ branching ratio. Therefore, It is an excellent channel to perform \kl measurement. \\
Photon objects are so crucial for this channel that work concerning the improvement of their identification efficiency is presented in this thesis. A new algorithm based on Convolutional Neural Network (CNN) is implemented to improve identification efficiency. The CNN algorithm uses images from the electromagnetic calorimeter layers to learn the shower development of prompt photons and separates them from photons originating from jet fragmentation or fake photons. In addition, the mis-modelling of the shower shape in the simulation is corrected by redistributing the energy between the cluster cells to become consistent with the real data. \\
The sensitivity to the \HHyybb signal is improved using a specific $b$-jet calibration technique. The signal is separated from the single Higgs and the continuum $\gamma\gamma$+jets backgrounds through a Multivariate Analysis (MVA). Additional tools are developed to improve the analysis sensitivity. They are not included in the publication of ATLAS collaboration presented in this thesis, but they would be used for the next analysis. No significant excess of \HHyybb events is observed in the data with respect to the expected background. Thus, an upper limit on the HH cross-section is set. The observed (expected) 95\% confidence level (CL) limit corresponds to 4.1 (5.5) times the cross-section predicted by the SM. The observed (expected) 95\% CL limit on the \kl modifier is also derived to be [-1.5, 6.7] ([-2.4, 7.7]).  \\
An extrapolation of this result to Run-3 and to High-Luminosity LHC (HL-LHC) is also presented in this thesis.

\chapter*{Résumé}
\addcontentsline{toc}{chapter}{Résumé}
Depuis la découverte du boson de Higgs en 2012, la plupart de ses propriétés telles que sa masse, son spin, sa section efficace de production et son couplage aux fermions et bosons ont été mesurées. Cependant, la composante trilinéaire de l'auto-couplage du boson de Higgs $\lambda_{HHH}$ n'a pas encore été mesuré. Ce paramètre contrôle la forme du potentiel de Higgs, donnant de l'importance à sa mesure. Une déviation par rapport à la valeur prédite par le Modèle Standard (MS) indiquerait la présence d'une nouvelle physique au-delà du MS. Les déviations sont quantifiées via le modificateur \kl. Au LHC, il est mesuré à travers le taux de production de paires de boson de Higgs (HH) qui est un processus rare et aussi le seul moyen direct d'y accéder. Cette thèse présente la recherche de la production de paires de boson de Higgs dans le canal de désintégration $H(\to\gamma\gamma)H(\to\bar{b}b)$. 
Le travail est effectué avec les données de collisions de protons du Run-2 enregistrées à une énergie de $\sqrt{s} = $ 13 TeV par le détecteur ATLAS au LHC entre 2015 et 2018. Cette quantité de données correspond à une luminosité intégrée de 139 \ifb. Malgré le faible nombre d'événements signaux, le canal \bbyy bénéficie d'une signature expérimentale très propre grâce d'une excellente résolution de masse \myy, d'un fond lisse et d'un grand rapport de branchement H$\to\bar{b}b$. C'est un canal excellent pour effectuer la mesure de \kl. \\
Les photons sont tellement cruciaux pour ce canal qu'un travail d'amélioration de leur identification est présenté dans cette thèse. Un nouvel algorithme basé sur les réseaux de neurones à convolution (CNN) est implémenté pour améliorer l'efficacité de l'identification. L'algorithme CNN utilise des images des cellules du calorimètre électromagnétique pour apprendre à reconnaitre les gerbes électromagnétiques des photons et les séparer des photons provenant de la fragmentation des jets ou des faux photons. De plus, la modélisation du développement des gerbes électromagnétiques en simulation est corrigée en répartissant l'énergie entre les cellules du calorimètre électromagnétique pour se rapprocher des données réelles. \\
La sensibilité au signal \HHyybb est améliorée en appliquant une méthode d'étalonnage spécifique aux jets de $b$. Le signal est ensuite séparé des bruits de fonds (boson de Higgs et continus $\gamma\gamma$+jets) grâce à une technique d'analyse multivariée (MVA). Des outils supplémentaires sont développés pour améliorer la sensibilité de l'analyse. Ils ne sont pas inclus dans la publication de la collaboration ATLAS, mais sont disponible pour la prochaine analyse.  Aucun excès significatif d'événements \HHyybb n'est observé dans les données par rapport au bruit de fond attendu. Ainsi, une limite supérieure sur la section efficace HH est fixée. La limite observée (attendue) à un niveau de confiance (CL) de 95 \% correspond à 4,1 (5,5) fois la section efficace prévue par le MS. La limite de 95 \% CL observée (attendue) sur le modificateur \kl est égale à [-1,5 ; 6,7] ([-2,4 ; 7,7]). \\
Une extrapolation de ce résultat au Run-3 et au LHC Haute-Luminosité (HL-LHC) est également présentée dans cette thèse.