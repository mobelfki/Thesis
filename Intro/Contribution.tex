\newpage
\chapter*{Personal contributions}

The work presented in this thesis was performed within the ATLAS collaboration. Even though this thesis will focus on my work, all the work described would not have been possible without major contributions from many other people and common tools and frameworks. In the following section, I list my contributions to the analysis and ATLAS  developments.

\section*{Selection of photons in ATLAS}

Chapter \ref{gamma} details the study of photon objects in ATLAS starting from reconstruction, calibration, isolation and identification. My contributions are the following:

\begin{enumerate}
    \item EM objects shower shape mis-modelling: (ATLAS qualification task) 
\begin{itemize}
    \item I implemented the developed (for electrons) shower shape reweighting technique to photon objects, and showed that the function developed for electron, was not enough to correct the mis-modelling for photons. 
    \item Therefore, I developed a specific reweighting function for photons and demonstrated that the reweighting procedure is not enough to correct the mis-modelling. 
    \item I provided a new idea for a three-dimension reweighting function and demonstrated its power using pseudo data.
    \item This work is documented in an internal note.
 
\end{itemize}

\item Photon identification: 
\begin{itemize}
    \item I developed a new photon identification algorithm based on Convolutional Neural Network (CNN).
    \item I quantified the improvement of the new algorithm in different event topologies. 
    \item I checked the performance of the CNN photon ID algorithm on 2017 data and evaluated the efficiency using the Radiative Z method. 
    \item I computed the scale factors using the 2017 data. 
    \item I presented ideas for possible improvements of the CNN algorithm.
\end{itemize}
\end{enumerate}

\section*{Jet reconstruction and Tagging in ATLAS}
Chapter \ref{Jet} details the jet reconstruction and calibration procedures as well as the different tagging algorithms. In this chapter, my contributions were mainly focusing on the $b$-jet calibration method in which: 
\begin{itemize}
    \item I optimized the muon selection for the $\mu$-in-jet correction.
    \item I derived the $p_T$-Reco correction function using $t\bar{t}$ sample for both reconstruction algorithms (Topo and PFlow) and both $b$-tagging algorithms (MV2c10 and DL1r). 
    \item I developed an analysis tool, that applies the $\mu$-in-jet and $p_T$-Reco corrections to the $b$-jet object.
\end{itemize}

\section*{Measurement of Higgs boson self-coupling}

Chapter \ref{HHyybb} focuses on the measurement of Higgs boson self-coupling. My contribution to this analysis are the following: 

\begin{itemize}
    \item I studied the improvement of the $b$-jet calibration on the analysis sensitivity.  
    \item I supervised an M1 student to work on the estimation of the systematic uncertainties related to the $b$-jet calibration. 
    \item I developed the likelihood fit (Kinematic Fit) and studied its impact on the \mbb resolution and analysis sensitivity. 
    \item I developed a tool interfacing the kinematic fit to the \bbyy event.
    \item I integrated both the $b$-jet and the kinematic fit tools in the ATLAS software used for the analysis (HGamCore).
    \item I participated in the ntuples (MxAOD) production for the analysis group.
    \item I developed the analysis strategy using the Deep Neural Network (DNN).
    \item I estimated the impact of the CNN photon identification on \HHyybb analysis.
    \item I contribute to the internal note documentation.
\end{itemize}

Chapter \ref{HL-LHC} presents an extrapolation of the measurement of the Higgs boson self-coupling performed in Chapter \ref{HHyybb}. Extrapolations to Run-3 and to HL-LHC are performed by me.

Plots and tables with caption in the box,
\begin{tcolorbox}[colback=black!5!white,colframe=white!75!black]
\end{tcolorbox}
are my personal contribution and produced by my self.