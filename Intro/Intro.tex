\chapter{Introduction}
\label{Intro}
The Standard Model (SM), developed in the early 1970s, encapsulates our best understanding of the fundamental particles and their interactions. It has demonstrated huge successes in providing experimental predictions which have been confirmed by the experimental observations over-time. The discovery of the Higgs boson by the ATLAS and CMS experiments in 2012 at the Large Hadron Collider (LHC) was a breakthrough for the experimental tests of the SM. Currently, precise measurements of the Higgs boson couplings to fermions and bosons, mass and cross-section are performed with the Run-1 and Run-2 datasets collected with both ATLAS and CMS detectors. The triple Higgs boson self-coupling, $\lambda_{HHH}$, still resists physicists. The $\lambda_{HHH}$ is present in the SM. This parameter controls the shape of the Higgs potential. The only direct way to measure this coupling is through the Higgs boson pair production process (HH). This process is mainly produced at the LHC via gluon-gluon fusion (ggF) through destructive interference of two Feynman diagrams involving quark loops and the triple Higgs boson self-interaction. At the LHC centre-of-mass energy of 13 TeV, the cross-section of the Higgs boson pair production is $31.05_{-5.0\%}^{+2.2\%}$ fb which is 1000 smaller than the single Higgs boson production cross-section. This small cross-section requires large integrated luminosity to make the first observation which will be delivered at HL-LHC. Since the Higgs boson is not a stable particle, the di-Higgs boson event is reconstructed through Higgs boson decay products. The HH decay channels are the combination of single Higgs boson decay channels. At LHC, the most sensitive decay channels in order of expected significance are HH$\to$\bbtt, HH$\to$\bbyy,  HH$\to$\bbbb, and HH$\to\bar{b}bVV$ ($V=W^{\pm},\ Z^0$). \\
If SM expectation hold, the observation of Higgs boson pair production is not possible with the currently available data, while many Beyond Standard Model (BSM) hypotheses predict enhancements of its cross-section either through new resonances (resonant production) or non-resonant production. The non-resonant production assumes new physics anomalies that predict a different value for the Higgs boson self-coupling. This deviation from the SM predicted value is quantified with $\kappa_{\lambda} = \frac{\lambda^{\text{new physics}}}{\lambda^{\text{SM}}}$ modifier, where $\lambda^{SM} = $ 0.13 is the predicted Standard Model $\lambda_{HHH}$ value. Searches for non-resonant and resonant di-Higgs boson production were performed by the ATLAS and CMS collaborations in several decay channels using about 36 \ifb of $pp$ collisions data from 2015-2016 at a centre-of-mass energy of 13 TeV. The ATLAS statistical combination of the \bbbb, \bbtt and \bbyy channels set an observed (expected) upper limit on the non-resonant HH production cross-section of around 6.9 (10) times the SM expectation and a constraint of the \kl parameters of [-5, 12] ([-5.8, 12]) at 95\% C.L. The CMS statistical combination of the same channels set an observed (expected) upper limit on the non-resonant HH production cross-section of around 22.2 (12.8) times the SM expectation and a constraint on \kl of [-11.8, 18.8] ([-7.1, 13.6]) at 95\% C.L. These results show that the sensitivity to this process is still limited with statistics used, making it a flagship analysis for the HL-LHC. It is nevertheless important to continue exploring HH production with the increase of the luminosity in data 2015-2018 (139 \ifb) and to improve analysis techniques for this process. \\
This thesis focuses on the search for the non-resonant Higgs boson pair production in the \bbyy final state, exploring the decay where one Higgs boson decays to two $b$-quark and the other to two photons. This search is translated into the measurement of the self-coupling quantifier \kl using the full Run-2 data collected by ATLAS detector. The branching ratio of this decay channel is only 0.3\%, but still considered as a "golden" channel thanks to the well-measured photon energy. This channel was already explored in early Run-2 using 36 \ifb of $pp$ collisions. The full Run-2 analysis aims to improve the 36 \ifb results on top of the luminosity increase by implementing new analysis techniques.\\
This thesis is structured as follows: Chapter \ref{chap1} presents an overview of the SM and its particles, the Brout-Englert-Higgs mechanism, Higgs boson production, its decay at LHC and summary of the Higgs boson discovery, as well as a summary of the Higgs boson pair production and decay channels and the 36 \ifb results. A description of the LHC and ATLAS detector, as well as the methods used to reconstruct the physics object, are presented in Chapter \ref{LHC&ATLAS}. To reconstruct the \HHyybb event an excellent knowledge of the photon, and therefore of the detector, is crucial. Chapter \ref{gamma} focus on the different methods used to reconstruct photon and a new photon identification algorithm is presented in this chapter. Similarly to photons, jets and $b$-jets are mandatory for \HHyybb analysis. Jets reconstruction and tagging as well as a specific calibration of the $b$-jet energy is presented in Chapter \ref{Jet}. The complete description of the \HHyybb analysis, including signal and background Monte Carlo simulations, objects and events selection, event categorization, background modelling, systematic uncertainties, statistical interpretation and results as well as comparison to the \HHyybb 36 \ifb results, \bbbb, \bbtt and the 139 \ifb CMS \HHyybb results, is given in Chapter \ref{HHyybb}. Finally, a description of the prospective for Run-3 analysis and the HL-LHC is given in Chapter \ref{HL-LHC}. 
