\newpage
\chapter{Conclusion}

In this thesis, a search for the Standard Model (SM) Higgs boson pair production and new physics phenomena in its decay to two photons and two $b$-quark channel is presented. The search is based on the full Run-2 dataset recorded by the ATLAS detector between 2015 and 2018 at a centre-of-mass energy of $\sqrt{s} = $ 13 TeV provided by the Large Hadron Collider (LHC) leading to a total of the integrated luminosity of 139 \ifb.  \\

The search is restricted to the non-resonant mode as a way to test the sensitivity to the SM process and new physics that could manifest as modifications of the Higgs boson self-coupling. Modifications of the Higgs boson self-coupling are quantified mainly through the \kl parameter. Despite the small branching ratio (0.3\%) of the \bbyy decay channel, it is a "golden" channel since it compromises between the H$\to\bar{b}b$ large branching ratio and a clean di-photon signature in the detector and its smoothly falling, continuous backgrounds. \\

To perform such analysis, an excellent identification of photon is needed. Such identification is currently performed on high-level shower shape variables. A new photon identification algorithm based on advanced Machine Learning (ML) technique is introduced. This algorithm is based on the Convolutional Neural Network (CNN) technique to learn prompt photon shower shape from electromagnetic calorimeter images. With this algorithm, a photon identification efficiency of 85-95\% is obtained. The algorithm is validated in different event topology and scale factors are extracted using radiative $\to\gamma l^-l^+$ method. In addition, a cell-by-cell reweighting technique is presented to correct the discrepancy between real data and simulated events. The sources of this discrepancy still remaining not clear, many sources can contribute to this mis-modelling, mainly from the detector geometry, material distribution and EM shower modelling. \\

As the $b$-jet energy is under-estimated which degrades \mbb resolution, excellent calibration of the $b$-jet energy scale and resolution is also needed for this search. A jet-by-jet calibration technique specific for $b$-jet is presented. The method attempts to correct the $b$-jet from the presence of muons and neutrinos in its semi-leptonic decay and the out-of-cone radiation effects. With this method, an improvement of 23\% in \mbb resolution is achieved and an enhancement of 10\% of the \HHyybb analysis sensitivity is obtained. The related systematic to this calibration method is also studied and found to be negligible. The $b$-jet calibration is complemented with a kinematic fit to constrain the di-Higgs system balance using the well-reconstructed H$\to\gamma\gamma$, which bring an additional 10\% improvement in \mbb resolution. \\

The results of this search are presented as an upper limit on the cross-section and as upper limits on the cross-section as a function of the \kl modifier to set a constraint on it. The observed (expected) upper limit on the Higgs pair production cross-section of 4.1 (5.5) times the SM expectation which corresponds to an observed (expected) cross-section upper limit 140 (180) fb, assuming the SM couplings. The allowed observed (expected) interval at 95\% CL obtained with this search is -1.5 $<$ \kl $<$ 6.7 (-2.4 $<$ \kl $<$ 7.7). They are the best upper limit on the Higgs pair production and \kl constraint up to date. This search is improved with a factor of 3 on top of the luminosity increase with respect to the previous search performed on subsets of the Run-2 dataset of 36\ifb. The improvement of the sensitivity of the search presented in this thesis is the result of the use of MultiVariate Analysis (MVA) techniques to target the HH signal and the improvement of the $b$-jet energy resolution as well as the categorization of the event over the $m_{HH}$ invariant distribution which enhances the sensitivity to \kl measurement. The results are compared to the one obtained by CMS collaboration in the same decay channel with the same amount of data. CMS reported an observed (expected) upper limit on the SM HH production of 7.7 (5.2) and an observed (expected) 95\% CL constrain on \kl parameter of -3.3 $<$ \kl $<$ 8.5 (-2.5 $<$ \kl $<$ 8.2). One should mention that the results are statistically dominated and the effect of systematic uncertainties is around 2\%. \\

Since ATLAS and CMS have similar results, an expected upper limit and \kl constraint from the statistical combination of the ATLAS and CMS full Run-2 \HHyybb results is estimated by extrapolating the presented results in the thesis to an integrated luminosity of 278\ifb which is twice the luminosity used to perform the results. An improvement of 1.4 on the upper limit on the SM HH production cross-section is expected from the extrapolation. The expected allowed 95\% CL interval of \kl from the extrapolation is -1.2 $<$ \kl $<$ 6.6. Moreover, prospect studies of the \HHyybb search at the Run-3 and HL-LHC are performed in this thesis. The extrapolation to Run-2+Run-3 analysis is done by assuming an additional 300\ifb at $\sqrt{s} = $ 13 TeV. This extrapolation leads to an expected upper limit on the SM HH production cross-section of 2.6 times SM expectations without any systematics and an expected Confidence Interval (CI) of \kl at 95\% of [-1.5, 6.6]. Further improvements of the sensitivity are possible to boost the \HHyybb analysis by using more advanced techniques for events reconstruction and classification either presented in this thesis or will come over years of R\&D. At the end HL-LHC runs, a total of 3000\ifb of data at $\sqrt{s} = $ 14 TeV is expected to be collected by ATLAS detector. The \HHyybb results are extrapolated to the HL-LHC integrated luminosity while neglecting the impact of the centre-of-mass energy. The expected \kl CI at 95\% from the extrapolation to HL-LHC is [-0.1, 4.7]. At the end of HL-LHC, the signal strength is expected to be measured with an accuracy of 28\% assuming no systematic uncertainties. The extrapolation to HL-LHC is compared to the current published combined extrapolation of \bbbb, \bbtt and \bbyy channels from the 27-36\ifb results. The extrapolation of the full Run-2 \HHyybb shows similar sensitivity to the combined extrapolation. The expected significance of observing the SM Higgs pair production at the end of HL-LHC in its \bbyy decay channel is 2.3$\sigma$. In combination with other channels and the CMS results, the Higgs pair production can be observed with a significance of around 4.5$\sigma$. 