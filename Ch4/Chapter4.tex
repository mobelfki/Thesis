\newpage
\chapter{Photons in ATLAS}
\label{gamma}

This include :
\begin{itemize}
    \item Introduction to electromagnetic objects and shower 
    \item Photon reconstruction 
    \item Photon Isolation 
    \item Photon identification (Run 2 cut based)
    \item Shower shape mis-modelling 
    \item Convolutional Neural Network for photon identification
\end{itemize}

\section{Photons Reconstruction}
\label{gamma:PR}

In this section, I will describe in details the reconstruction of photons and energy calibration.

\section{Photons Isolation}
\label{gamma:Iso}

Photon isolation and different isolation WPs will be discussed here.

\section{Photons Identification}
\label{gamma:ID}

The identification algorithm will be discussed here, and how the shower shape mis-modelling is affecting the ID efficiency.

\section{Shower shape mis-modelling}
\label{gamma:ss}

The QT work will fit here. 

\section{Convolutional Neural Network for Photon Identification}
\label{gamma:CNN}

The CNN work will fit here to improve the photon identification efficiency and if scale factor are ready by the date of the thesis will be included to.

\section{Conclusion}
\label{gamma:conc}

A conclusion to the chapter will be discussed here by answering to :

\begin{itemize}
    \item What is the limits for the current ID method (cut based) ?
    \item How CNN is improving the ID efficiency ?
    \item What is the limits of the CNN model ?
    \item How we can improve the CNN for future ?
\end{itemize}