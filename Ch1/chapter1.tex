\newpage
\chapter{Theoretical Framework}
\label{chap1}

This chapter should contains all theory needed for SM:
\begin{itemize}
    \item Describe the elementary particles 
    \item Describe the fundamental interactions and their related fields
    \item Describe the EWSB and Higgs physics introduction
    \item Brief introduction to p-p collision and higgs production at LHC
    \item Discovery of Higgs boson at CERN 2012
    \item Higgs self coupling anomaly
    \item Describe Di-Higgs production as a direct process to measure higgs self coupling and probe of BSM effects
    \item Early Run 2 results and limits on \kl
    \item Introduction to EFT as an alternative interpretation of Full Run 2 results to probe physics at large scale 
\end{itemize}

\section{Introduction}
\label{chap1:intro}

This section should include a brief history of particle physics and the SM. \\

How the world around us made? How does it work? These are questions asked a long-ago to understand the Universe. The first efforts to elucidate these question referred to ancient Greek philosophers. Greeks gave much to the physics domain by developing the fundamental basis of modern principles as the conservation of matter, atomic theory. Democritus's model introduces Atom as small indivisible building blocks (particle) that matter consists off. At that time, atoms allowed to describe a variety of phenomena. Rutherford comes in 1909 with his experience says that atoms consist of mostly space with Electrons surrounding a dense central nucleus made up of Protons and Neutrons. At that time, Newtonian laws of motion and atoms consist off a solid framework. Currently, the Standard Model (SM) is the most accurate model describing the universe composition. In the SM, there are two types of elementary particles: fundamental constituents of matter called "Fermions", and the quanta of fields called "Bosons" exchanged between when an interaction occurs between fermions. The SM has been successfully, so far, at predicting the results from the measurements performed in the past 50. The following sections provide more details about the standard model and its particles.

\section{Standard Model (SM) of particle physics}
\label{chap1:SM}
A possible goal of physicists is to reduce all natural phenomena to a set of fundamental laws and theories which, at least in principle, can quantitatively explain and predict experimental results. At the microscopic scale, all matter behaviour and phenomenology, including molecular, atomic, nuclear, and subnuclear physics, can be explained under three fundamental interactions: electromagnetic, weak and strong forces. At the macroscopic scale, the fourth force, the gravitational interaction, has an essential role, while it is negligible at the microscopic scale. All the three interactions are describes within a local relativistic quantum field theory based on the principal gauge invariance. This is called "The Standard Model". In this model, elementary building blocks (particles) are divided into two types Fermions or Bosons. Within SM matter consists of fermions and interactions are mediated by bosons. 

\subsection{Elementary Particles}
\label{chap1:SM:EP}

This should include a description of elementary particles, Fermions and  Bosons and their properties (mass, spin, charge, quantum number)
\\
\\
In the SM, particles classified as either fermions or bosons depending on what statistics they obey. Fermions obey Fermi-Dirac statistics and respect the Pauli exclusion principle, i.e. two fermions in the same quantum state can not exist in the same place and time. Such particles have an intrinsic angular momentum, called spin J, half-integer. Bosons obey Bose-Einstein statistics, due to spin-statistics theorem, they have integer spin value. Through an interaction, a boson emitted by a matter particle and then absorbed by another particle. Fermions divided into two categories: Leptons and Quarks.
\subsubsection{Leptons}
Leptons (comes from the Greek word meaning "light") are grouped into three families or generations formed by three charged leptons: electron e, muons $\mu$ and tau $\tau$ with electric charge -e, with e being the elementary charge of $\sim 1.6 \ 10^{-19} $C, and their neutral complemented partners neutrinos : $\nu_{e}, \nu_{\mu}$ and $\nu_{\tau}$. Only electron and neutrinos (in SM) are stable. A quantum number called Leptonic number (L) is associated with each lepton. Electron, muon and tau have identical properties (a charge, spin) however tau is 3477 times heavier than an electron, the muon has 17 times the mass of an electron. The rest mass of an electron is $9.10938356 \ 10^{-31} Kg$. From 20 May 2019  kilogram not anymore part of the International System of Units for that masses will be expressed by the unit of energy (E) the electron-volt (eV). Mass is related to eV by the equation $E=mc^2$, c is the velocity of light. In particle and high energy physics, all constants fixed to one (c=1). Electron mass is $511 \ keV$.
\subsubsection{Quarks}
Quarks are electrically charger particles, with charge of $+\frac{2}{3}e$ for so-called up-type quarks and $-\frac{1}{3}e$ for the down-type quarks. There are sin known quarks, similarly to leptons quarks are paired into three generations. The first generation consists of up (u) and down (d) quarks, the second has the charm (c) and strange (s) quarks and top (t) and bottom (b) quarks for the third generation. Quarks can not exist in a free state. Table \ref{tab:fermions} shows a summary of leptons and quarks. \\
Each of the higher generations has particles with higher mass and tends to decay to the lower one, explains why the ordinary matter made off the first-generation particles. Each fermion has it is corresponding anti-particle.
\begin{table}[ht]
\centering
\begin{tabular}{c|c|c|c|}
\cline{2-4}
 &
  1st Generation &
  2nd Generation &
  3rd Generation \\ \hline
\multicolumn{1}{|c|}{Quarks} &
  \begin{tabular}[c]{@{}c@{}}u\\ 2.16 MeV\\ $+\frac{2}{3}$\end{tabular} &
  \begin{tabular}[c]{@{}c@{}}c\\ 1.27 GeV\\ $+\frac{2}{3}$\end{tabular} &
  \begin{tabular}[c]{@{}c@{}}t\\ 172.4 GeV\\ $+\frac{2}{3}$\end{tabular} \\ \cline{2-4} 
\multicolumn{1}{|c|}{} &
  \begin{tabular}[c]{@{}c@{}}d\\ 4.67 MeV\\ $-\frac{1}{3}$\end{tabular} &
  \begin{tabular}[c]{@{}c@{}}s\\ 93 MeV\\ $-\frac{1}{3}$\end{tabular} &
  \begin{tabular}[c]{@{}c@{}}b\\ 4.18 GeV\\ $-\frac{1}{3}$\end{tabular} \\ \hline
\multicolumn{1}{|c|}{Leptons} &
  \begin{tabular}[c]{@{}c@{}}e\\ 0.511 MeV\\ -1\end{tabular} &
  \begin{tabular}[c]{@{}c@{}}$\mu$\\ 105.7 MeV\\ -1\end{tabular} &
  \begin{tabular}[c]{@{}c@{}}$\tau$\\ 1.8 GeV\\ -1\end{tabular} \\ \cline{2-4} 
\multicolumn{1}{|c|}{} &
  \begin{tabular}[c]{@{}c@{}}$\nu_{e}$\\ 0\\ 0\end{tabular} &
  \begin{tabular}[c]{@{}c@{}}$\nu_{\mu}$\\ 0\\ 0\end{tabular} &
  \begin{tabular}[c]{@{}c@{}}$\nu_{\tau}$\\ 0\\ 0\end{tabular} \\ \hline
\end{tabular}
\caption{Generations of quarks and leptons with their masses and charges}\label{tab:fermions}
\end{table}
Standard Model of elementary particles assumes neutrinos to be mass-less particles, while some experiments demonstrate that neutrinos have a non-negligible mass $\sim eV$. 

\subsubsection{Bosons}
As mentioned before, bosons are particles of integer angular momentum and obey to the Bose-Einstein statistics. They are the carriers of the gauge interactions between fermions. Photon ($\gamma$) is a boson known as a quantum of the electromagnetic field including electromagnetic radiation such as light. Photons are neutral and mass-less particles.  $W^{\pm}, Z^{0}$ are bosonic particles wish carriers the weak interaction. $Z^{0}$ boson is neutral while $W^{\pm}$ charged with a charge of $\pm$e. Contrary to photons, Weak bosons are massive bosons. $W^{\pm}$ and $Z^{0}$ mess predicted to be so large that it took many years to build powerful accelerators to produce them. $W^{\pm}$ and $Z^{0}$ bosons have been discovered at CERN in 1983 by the UA1 and UA2 collaboration, and their masses were found to be about 80 GeV and 91 GeV respectively. Gluons are the neutral quantum of the strong force known as the "glue" that links quarks to form hadrons. The mass of the gluons is known to be strictly zero. They are eights gluons differed with their colours charge. 
\subsection{Elementary Interactions}
\label{chap1:SM:EI}

This should include a description of the 4 interaction and their properties.\\
The associated group for each of the SM interactions (U(1) SU(2) SU(3)) and the its Lagrangian.\\
How SM combines the 3 groups in one fundamental theory which describe particle physics.\\
Brief summary of boson vs interactions

There are three conventionally taken fundamental forces (other than gravitation) electromagnetic, weak and strong. The interactions relate to matter (fermions) by the transmission of a boson. As mentioned above SM particle content and interactions can be expressed more formally through the concepts of symmetries and gauge invariance. The generators of this group correspond to the gauge bosons that are mediators of a fundamental force and responsible for the interactions. Electromagnetic interaction mediated by photons. Weak forces used $W^{\pm}$ and $Z^{0}$ as mediators while the Gluons are mediators for the strong interaction.
\subsubsection{Electromagnetic interaction}
Electromagnetic interaction describes the dynamics of charged fermions. It is described by the Quantum Electrodynamics (QED). Each quantum field theory is represented by a Lagrangian density, the QED Lagrangian representing the behaviour of a freely propagating fermion field $\psi (x,t)$ can be written as : 
\begin{equation}
    \mathcal{L} = \bar{\psi}i\gamma^\mu\partial_\mu\psi - m\psi\bar{\psi},
\end{equation}
where m is the mass of the particle. The Einstein convention is used here, with the indices $\mu= 0,1,2,3$ representing the space-time components x and t. \\ 
To be a valid gauge theory the QED lagrangian should invariant under a U(1) local gauge transformation of the field : $\psi\rightarrow e^{i\alpha(x)}\psi$. This condition leads to additional terms to be added to the lagrangian and new gauge field $A_{\mu}$ that represents the photon:
\begin{equation}
    \mathcal{L} = \bar{\psi}i\gamma^\mu\partial_\mu\psi - m\psi\bar{\psi} + q\psi\gamma^{\mu}\psi A_{\mu} - \frac{1}{4}F^{\mu\nu}F_{\mu\nu},
\end{equation}
where $F^{\mu\nu} = - F^{\mu\nu} = \partial^{\nu}A^{\mu} - \partial^{\mu}A^{\nu}$ is the field-strength tensor for the electromagnetic force, as described by Maxwell equations, describes the kinetic propagation of the field. The term $q\psi\gamma^{\mu}\psi A_{\mu}$ reflects the interaction between a fermion and the electromagnetic force (photon) with strength q (q=-e) the electromagnetic interaction charge (electric charge). The mass term of the photon is not added to the lagrangian since it spoils the gauge invariance. This is in agreement with the observation that the photon is mass-less. U(1) has one generator corresponds to the photon mediator.

\subsubsection{Electro-Weak interaction}


\section{Electroweak Symmetry Breaking (EWSB)}

\label{chap1:EWSB}

This section will ask the question mass of particles and introduce the EWSB to answer.\\
Introduce the Higgs boson and its proprieties.\\

\subsection{Higgs boson production}
\label{chap1:EWSB:HP}

Higgs boson production modes ggF VBF VH and qqH

\subsection{Higgs boson decay channels}
\label{chap1:EWSB:HD}

Higgs boson decay modes and their rates.

\section{Higgs Boson Discovery}
\label{chap1:H2012}

This should contains the story of Higgs boson discovery by ATLAS and CMS in 2012.

\subsection{p-p collisions}
\label{chap1:H2012:PP}

The protons protons collision and the Higgs boson cross section at 8 TeV and 13 TeV.

\subsection{Higgs measurements}
\label{chap1:H2012:HM}

The discovery of Higgs boson in 2012 by ATLAS and CMS and first measurements of its mass.\\
Recap of Higgs boson proprieties. and Introduction to self couplings no yet measured.

\section{Double Higgs boson}
\label{chap1:HH}

This should contain the theory of HH and how it allows to measure the higgs self couplings.

\subsection{Di-Higgs production and decays}
\label{chap1:HH:HPD}

This should contain the production modes and decays of HH. and the \HHyybb channel.

\subsection{Di-Higgs as a probe of BSM physics}
\label{chap1:HH:BSM}
How \kl cloud probe the existence of BSM effects

\subsection{Current Measurements}
\label{chap1:HH:CM}
This includes the 36 \ifb results and the limit on HH cross section and \kl.

\section{Effective Field Theory (EFT)}
\label{chap1:EFT}

A brief introduction to EFT and how it helps to probe physics at large scale.\\

\section{Conclusion}
\label{chap1:Conc}

Brief conclusion of chapter 1.




